\documentclass[12pt]{report}
\usepackage[brazil]{babel}
\usepackage[utf8]{inputenc}
\usepackage{amsbsy,amsfonts,amsmath,amssymb,amsthm}
\usepackage{float}
\usepackage{graphicx}

\begin{document}
\begin{itemize}
    \item Slide 22
\end{itemize}

Quando $n$ tende ao infinito com $E(X_{1}^{4}) < \infty$, podemos analisar o comportamento da expressão $S^{2}$:

$$S^{2} = \frac{n}{n-1} \left[ \frac{1}{n} \sum_{i=1}^{n} x_{i}^{2} - \bar{x}^{2} \right]$$

Primeiro, vamos observar os termos da expressão:
\begin{enumerate}

\item $\frac{1}{n} \sum_{i=1}^{n} x_{i}^{2}$ representa a média dos quadrados dos elementos da amostra ($x_{i}$).

\item $\bar{x}^{2}$ representa a média amostral ($\bar{x}$) dos elementos da amostra ao quadrado.

Aqui, estamos interessados em como essa expressão se comporta quando $n$ tende ao infinito e $E(X_{1}^{4}) < \infty$.

O termo $\frac{1}{n} \sum_{i=1}^{n} x_{i}^{2}$ se refere à média dos quadrados dos elementos da amostra. Conforme $n$ aumenta, a média dos quadrados tende a se aproximar do quarto momento ($E(X_{1}^{4})$) da distribuição, já que a lei dos grandes números sugere que a média amostral se aproxima do valor esperado.

O termo $\bar{x}^{2}$ representa a média amostral dos elementos da amostra ao quadrado. Assim como explicado anteriormente, à medida que $n$ tende ao infinito, a média amostral se aproxima do valor esperado ($E(X)$), e portanto, $\bar{x}^{2}$ se aproxima de $E(X)^{2}$.

O fator $\frac{n}{n-1}$ na frente da expressão é uma correção para a estimativa de variância quando se trabalha com amostras (em oposição à população completa). Quando $n$ é grande, a diferença entre $n$ e $n-1$ se torna menos significativa, e essa correção tende a 1.

Portanto, sob as condições em que $E(X_{1}^{4}) < \infty$ e quando $n$ tende ao infinito, a expressão $S^{2}$ tende a se aproximar do valor esperado da quarta potência da variável aleatória ($E(X_{1}^{4})$) ou do quarto momento. Em outras palavras, $S^{2}$ se aproxima da variabilidade intrínseca da distribuição, refletindo a dispersão dos valores ao redor do valor esperado.
\end{enumerate}
\end{document}