\documentclass[14pt,aspectratio=1610]{beamer}

\usepackage[brazil]{babel}
\usepackage[utf8]{inputenc}
%\UseRawInputEncoding
\usepackage[T1]{fontenc}
%\usepackage{Sweave}
\usepackage{animate}
\usepackage{amsbsy}
\usepackage{amsfonts}
\usepackage{amsmath}
\usepackage{amssymb}
\usepackage{amsthm}
\usepackage[toc,page,title,titletoc]{appendix}
%\usepackage[fixlanguage]{babelbib}
%\usepackage[pdftex]{color}
\usepackage{dsfont}
\usepackage{esvect}
\usepackage[labelfont=bf]{caption}
\usepackage{subcaption}
\usepackage{float}
\usepackage[Glenn]{fncychap}%Sonny %Conny %Lenny %Glenn %Renje %Bjarne %Bjornstrup
%\usepackage{geometry, calc, color, setspace}%
%\geometry{a4paper, headsep=1.0cm, footskip=1cm, lmargin=3cm, rmargin=2cm, tmargin=3cm, bmargin=2cm}
\usepackage{graphicx}
\usepackage{indentfirst}%Para indentar os parágrafos automáticamente
\usepackage{lipsum}
\usepackage{longtable}
\usepackage{mathtools}
\usepackage{listings}%Inserir codigo do R no latex
%\usepackage{slashbox}
\usepackage{multirow}
\usepackage{multicol}
\usepackage{csquotes}
\usepackage[maxcitenames=2,terseinits=true,natbib=true, style=authoryear, maxbibnames=99]{biblatex}
\addbibresource{../Referencias/Referencias.bib}
%\usepackage{csquotes}
%\usepackage[natbib=true,style=abnt, sorting=none]{biblatex}
%\addbibresource{bibliografia.bib}
\usepackage[figuresright]{rotating}
\usepackage{spalign}
%\usepackage{pgfpages}
\usepackage{pgfplots}
\pgfplotsset{compat=1.18}
\usepackage{tikz}
\usepackage{color, colortbl}
\usepackage{ragged2e}%para justificar o texto dentro de algum ambiente
\definecolor{Gray}{gray}{0.9}
\definecolor{LightCyan}{rgb}{0.88,1,1}
\definecolor{Lightblue}{RGB}{50, 149, 168}
%\usepackage{grffile}

\usepackage[all]{xy}



%\usetheme{Madrid}
%\usecolortheme[RGB={193,0,0}]{structure}
\usetheme{metropolis}
\definecolor{mycolor}{RGB}{34, 45, 50}
\setbeamercolor{structure}{fg=mycolor}
\usepackage{mathpazo} % Fonte elegante para matemática
\usepackage{helvet} % Fonte sans-serif para texto
\renewcommand{\familydefault}{\sfdefault} % Definir fonte padrão como sans-serif

%\setbeamertemplate{footline}[frame number]
%\setbeamertemplate{footline}[text line]{%
%  \parbox{\linewidth}{\vspace*{-8pt}\hfill\date{}\hfill\insertshortauthor\hfill\insertpagenumber}}
\beamertemplatenavigationsymbolsempty
\renewcommand{\vec}[1]{\mbox{\boldmath$#1$}}
\newtheorem{Teorema}{Teorema}
\newtheorem{Proposicao}{Proposição}
\newtheorem{Definicao}{Definição}
\newtheorem{Corolario}{Corolário}
\newtheorem{Demonstracao}{Demonstração}
\newcommand{\bx}{\ensuremath{\bar{x}}}
\newcommand{\Ho}{\ensuremath{H_{0}}}
\newcommand{\Hi}{\ensuremath{H_{1}}}
\everymath{\displaystyle}

\apptocmd{\frame}{}{\justifying}{} % Allow optional arguments after frame.

\title{Estatística I}
\author{Prof. Fernando de Souza Bastos \texorpdfstring{\\ fernando.bastos@ufv.br}{}}
\institute{Departamento de Estatística \texorpdfstring{\\ Universidade Federal de Viçosa}{}\texorpdfstring{\\ Campus UFV - Viçosa}{}}
\date{}
\newcommand\mytext{Aula de Exercícios}
\newcommand\mytextt{Fernando de Souza Bastos}
\newcommand\mytexttt{\url{https://ufvest.github.io/}}

\makeatletter
\setbeamertemplate{footline}
{
  \leavevmode%
  \hbox{%
  \begin{beamercolorbox}[wd=.3\paperwidth,ht=2.25ex,dp=1ex,center]{author in head/foot}%
    \usebeamerfont{author in head/foot}\mytext
  \end{beamercolorbox}%
  \begin{beamercolorbox}[wd=.3\paperwidth,ht=2.25ex,dp=1ex,center]{title in head/foot}%
    \usebeamerfont{title in head/foot}\mytextt
  \end{beamercolorbox}%
  \begin{beamercolorbox}[wd=.35\paperwidth,ht=2.25ex,dp=1ex,right]{site in head/foot}%
    \usebeamerfont{site in head/foot}\mytexttt\hspace*{2em}
    \insertframenumber{} / \inserttotalframenumber\hspace*{2ex} 
  \end{beamercolorbox}}%
  \vskip0pt%
}
\makeatother

\providecommand{\arcsin}{} \renewcommand{\arcsin}{\hspace{2pt}\textrm{arcsen}}
\providecommand{\sin}{} \renewcommand{\sin}{\hspace{2pt}\textrm{sen}}
%\newtheorem{Teorema}{Teorema}
%\newtheorem{Proposicao}{Proposição}
%\newtheorem{Definicao}{Definição}
%\newtheorem{Corolario}{Corolário}
%\newtheorem{Demonstracao}{Demonstração}

\titlegraphic{\hspace*{8cm}\href{https://fsbmat-ufv.github.io/}{\includegraphics[width=2cm]{figs/mylogo.png}}
}


\usepackage{hyperref,bookmark}
\hypersetup{
  colorlinks=true,
  linkcolor=blue,
  citecolor=red,
  filecolor=blue,
  urlcolor=blue,
}

% Layout da pagina
\hypersetup{pdfpagelayout=SinglePage}
\begin{document}
%\input{Aula16-concordance}

\frame{\titlepage}


\begin{frame}{}
\frametitle{}
\begin{block}{}
\justifying
Sabe-se que 30\% dos indivíduos que recebem certo medicamento sofrem com alguns efeitos colaterais. Se esse medicamento for ministrado a cinco pacientes, qual a probabilidade de que no máximo 2 sofram efeitos colaterais?
\begin{enumerate}
\item Faça esse cálculo utilizando o modelo Binomial.
\item Faça esse cálculo utilizando o modelo de Poisson. (Lembre-se que $\lambda=np$)
\end{enumerate}

\end{block}
\end{frame}

\begin{frame}{}
\frametitle{}
\begin{block}{}
\justifying
\footnotesize
Uma empresa vende sementes de Jatobá (\emph{Hymenaea courbaril}) em pacotes com 5 unidades. Seja $X$ a variável aleatória discreta que informa o número de sementes que germinam por pacote, cuja distribuição de probabilidades é dada na tabela a seguir.
\begin{table}[!htbp]
\centering
\begin{tabular}{c|cccccc|c}
$x$&0&1&2&3&4&5&Total \\ \hline
$P\left[X=x\right]$&0,01&0,01&0,03&0,06&0,29&0,60&1,00\\
\end{tabular}
\end{table}

\begin{enumerate}
\item Se a firma vender 200 pacotes de sementes, em quantos destes pacotes espera-se que germinarão no máximo 2 sementes?
\item Se a firma garante que um número mínimo de $x_0$ sementes germinarão por pacote, qual deve ser este valor $x_0$ para que $95\%$ dos pacotes atendam à esta garantia?
\item Calcule a seguinte probabilidade condicional: $P \left[X=5\left.\right|3\leq X\leq5\right]$
\end{enumerate}
\end{block}
\end{frame}

\begin{frame}{}
\frametitle{}
\begin{block}{}
\justifying
\footnotesize
\begin{table}[!ht]
\centering
\renewcommand{\arraystretch}{1.8}
\begin{tabular}{c|cccccc}
\hline
		$x_i$ & 4 & 5 & 6 & 7 & 8 & 9 \\
\hline
		 $P\left[X=x_i\right]$ & $\frac{1}{12}$ & $\frac{1}{12}$ & $\frac{1}{4}$ & $\frac{1}{4}$ & $k$ & $k$ \\
\hline
\end{tabular}
\end{table}

\begin{enumerate}
\item O valor de $k$ para que a distribuição acima seja uma distribuição de probabilidades?
\item Se para $X$ carros lavados o funcionário tem um ganho de $2X-1$ reais, qual seria o seu ganho médio esperado?
\item Se, devido a problemas financeiros, o proprietário do lava a jato precisar reduzir em 10\% o ganho do funcionário por carro lavado, qual seria agora o seu ganho médio esperado?
\item  Enuncie pelo menos duas propriedades de variância, $V\left[X\right]$.
\end{enumerate}
\end{block}
\end{frame}

\begin{frame}{}
\frametitle{}
\begin{block}{}
\justifying
Um inspetor de qualidade extrai uma amostra de 10 tubos armazenados num depósito onde, de acordo com os padrões de produção, se espera um total de 20\% de tubos defeituosos. Pede-se:
\begin{enumerate}
\item Qual é a probabilidade de que não mais do que 2 tubos extraídos sejam defeituosos?
\item Quantos tubos defeituosos espera-se encontrar nessa amostra?
\item Cite as duas principais diferenças entre o modelo Binomial e o modelo de Poisson.
\end{enumerate}
\end{block}
\end{frame}

\begin{frame}{}
\frametitle{}
\begin{block}{}
\justifying
Suponha que, em média, uma delegacia de uma pequena cidade prenda 1 indivíduo por dia com sintomas de embriaguez. Qual a probabilidade de que nessa cidade:
\begin{enumerate}
\item\label{examp:Poi3a} em um certo dia sejam presos dois indivíduos com sintomas de embriaguez?
\item\label{examp:Poi3b} em um certo dia sejam presos mais que dois indivíduos com sintomas de embriaguez?
\item\label{examp:Poi3c} em quatro dias sejam presos pelo menos 3 indivíduos com sintomas de embriaguez?
\end{enumerate}
\end{block}
\end{frame}

\begin{frame}{}
\frametitle{}
\begin{block}{}
\justifying
Suponha que o tempo (em minutos) que um jogador de futebol tem a posse da bola durante os 90 minutos de jogo possa ser aproximado por uma distribui\c{c}\~{a}o de Poisson com m\'{e}dia de tr\^{e}s minutos. A probabilidade de que em um jogo com prorroga\c{c}\~{a}o (120 minutos) um jogador tenha a possa da bola por pelo menos quatro minutos \'{e}:
\begin{enumerate}
\begin{multicols}{3}
\item 0,3711;
\item 0,4335;
\item 0,3528;
\item 0,1906;
\item 0,0465;
\item\label{corr03} n.d.r.a.
\end{multicols}
\end{enumerate}
\end{block}
\nocite{roteiro}
\nocite{Morettin09, Apostila, eric, montgomery2016, meyer1982probabilidade, Bastos2025}
\end{frame}

\begin{frame}[allowframebreaks]
\frametitle{\bf Referências}
\printbibliography
\end{frame}

\end{document}
