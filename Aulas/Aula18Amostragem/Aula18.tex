\documentclass[14pt,aspectratio=1610]{beamer}

\usepackage[brazil]{babel}
\usepackage[utf8]{inputenc}
%\UseRawInputEncoding
\usepackage[T1]{fontenc}
%\usepackage{Sweave}
\usepackage{animate}
\usepackage{amsbsy}
\usepackage{amsfonts}
\usepackage{amsmath}
\usepackage{amssymb}
\usepackage{amsthm}
\usepackage[toc,page,title,titletoc]{appendix}
%\usepackage[fixlanguage]{babelbib}
%\usepackage[pdftex]{color}
\usepackage{dsfont}
\usepackage{esvect}
\usepackage[labelfont=bf]{caption}
\usepackage{subcaption}
\usepackage{float}
\usepackage{verbatim}
\usepackage[Glenn]{fncychap}%Sonny %Conny %Lenny %Glenn %Renje %Bjarne %Bjornstrup
%\usepackage{geometry, calc, color, setspace}%
%\geometry{a4paper, headsep=1.0cm, footskip=1cm, lmargin=3cm, rmargin=2cm, tmargin=3cm, bmargin=2cm}
\usepackage{graphicx}
\usepackage{indentfirst}%Para indentar os parágrafos automáticamente
\usepackage{lipsum}
\usepackage{longtable}
\usepackage{mathtools}
\usepackage{listings}%Inserir codigo do R no latex
%\usepackage{slashbox}
\usepackage{multirow}
\usepackage{multicol}
\usepackage{csquotes}
\usepackage[citestyle=authoryear,maxcitenames=2,terseinits=true,natbib=true, style=abnt, maxbibnames=99]{biblatex}
\addbibresource{Referencias/Referencias.bib}
\usepackage[figuresright]{rotating}
\usepackage{spalign}
%\usepackage{pgfpages}
\usepackage{pgfplots}
\pgfplotsset{compat=1.18}
\usepackage{tikz}
\usetikzlibrary{shapes.geometric, arrows}
\usepackage{color, colortbl}
\usepackage{ragged2e}%para justificar o texto dentro de algum ambiente
\definecolor{Gray}{gray}{0.9}
\definecolor{LightCyan}{rgb}{0.88,1,1}
\definecolor{Lightblue}{RGB}{50, 149, 168}
%\usepackage{grffile}

\usepackage[all]{xy}
\usepackage{pifont}
\usepackage{enumitem}


\usetheme{Madrid}
\usecolortheme[RGB={193,0,0}]{structure}

%\setbeamertemplate{footline}[frame number]
%\setbeamertemplate{footline}[text line]{%
%  \parbox{\linewidth}{\vspace*{-8pt}\hfill\date{}\hfill\insertshortauthor\hfill\insertpagenumber}}
\beamertemplatenavigationsymbolsempty
\renewcommand{\vec}[1]{\mbox{\boldmath$#1$}}
\newtheorem{Teorema}{Teorema}
\newtheorem{Proposicao}{Proposição}
\newtheorem{Definicao}{Definição}
\newtheorem{Corolario}{Corolário}
\newtheorem{Demonstracao}{Demonstração}
\newcommand{\bx}{\ensuremath{\bar{x}}}
\newcommand{\Ho}{\ensuremath{H_{0}}}
\newcommand{\Hi}{\ensuremath{H_{1}}}
\everymath{\displaystyle}

\apptocmd{\frame}{}{\justifying}{} % Allow optional arguments after frame.

\title{Estatística I}
\author{Prof. Fernando de Souza Bastos \texorpdfstring{\\ fernando.bastos@ufv.br}{}}
\institute{Departamento de Estatística \texorpdfstring{\\ Universidade Federal de Viçosa}{}\texorpdfstring{\\ Campus UFV - Viçosa}{}}
\date{}
\newcommand\mytext{Aula 18}
\newcommand\mytextt{Fernando de Souza Bastos}
\newcommand\mytexttt{\url{https://ufvest.github.io/}}

\makeatletter
\setbeamertemplate{footline}
{
  \leavevmode%
  \hbox{%
  \begin{beamercolorbox}[wd=.3\paperwidth,ht=2.25ex,dp=1ex,center]{author in head/foot}%
    \usebeamerfont{author in head/foot}\mytext
  \end{beamercolorbox}%
  \begin{beamercolorbox}[wd=.3\paperwidth,ht=2.25ex,dp=1ex,center]{title in head/foot}%
    \usebeamerfont{title in head/foot}\mytextt
  \end{beamercolorbox}%
  \begin{beamercolorbox}[wd=.35\paperwidth,ht=2.25ex,dp=1ex,right]{site in head/foot}%
    \usebeamerfont{site in head/foot}\mytexttt\hspace*{2em}
    \insertframenumber{} / \inserttotalframenumber\hspace*{2ex} 
  \end{beamercolorbox}}%
  \vskip0pt%
}
\makeatother

\providecommand{\arcsin}{} \renewcommand{\arcsin}{\hspace{2pt}\textrm{arcsen}}
\providecommand{\sin}{} \renewcommand{\sin}{\hspace{2pt}\textrm{sen}}
%\newtheorem{Teorema}{Teorema}
%\newtheorem{Proposicao}{Proposição}
%\newtheorem{Definicao}{Definição}
%\newtheorem{Corolario}{Corolário}
%\newtheorem{Demonstracao}{Demonstração}

\titlegraphic{\hspace*{8cm}\href{https://fsbmat-ufv.github.io/}{\includegraphics[width=2cm]{figs/mylogo.png}}
}


\usepackage{hyperref,bookmark}
\hypersetup{
  colorlinks=true,
  linkcolor=blue,
  citecolor=red,
  filecolor=blue,
  urlcolor=blue,
}

% Layout da pagina
\hypersetup{pdfpagelayout=SinglePage}
\begin{document}
%\input{Aula21-concordance}

\frame{\titlepage}

\begin{frame}{}
\frametitle{\bf Sumário}
\tableofcontents
\end{frame}

\section{Introdução}
\begin{frame}{}
    \begin{block}{}
    \justifying
A Estatística constitui-se numa excelente ferramenta quando existem problemas de variabilidade na produção. É uma ciência que trata da coleta e da interpretação de dados, ajudando no estabelecimento de conclusões e de normas sobre o problema estudado.
    \end{block}
\end{frame}

\begin{frame}{Conceitos Fundamentais}
    \begin{block}{}
    \justifying
\begin{itemize}[label={\ding{226}}]
\item \textbf{População:} conjunto de todos os elementos sob investigação.
\item \textbf{Amostra:} subconjunto da população.
\item \textbf{Variável de interesse:} característica a ser observada em cada indivíduo da amostra.
\end{itemize}
    \end{block}
\end{frame}

\begin{frame}{Etapas da Análise Estatística}
    \begin{block}{}
    \justifying
\begin{itemize}[label={\ding{228}}]
\item Definir a população de interesse.\pause
\item Estabelecer os objetivos da pesquisa.\pause
\begin{itemize}[label={\ding{226}}]
\item Definir critérios objetivos sobre quais dados coletar.\pause
\item Postular a análise estatística a ser utilizada.\pause
\end{itemize}
\item Definir o método para coletar as amostras.\pause
\begin{itemize}[label={\ding{226}}]
\item Fonte de dados secundários (IBGE, Portal da Transparência, Microdados do ENEM, etc);\pause
\item Banco de dados de uma empresa (observacional);\pause
\item Experimentos em laboratórios e/ou campo;\pause
\item Pesquisas amostrais.
\end{itemize}
\end{itemize}
    \end{block}
\end{frame}

\begin{frame}{Etapas da Análise Estatística}
    \begin{block}{}
    \justifying
\begin{itemize}[label={\ding{228}}]
\item Coletar e organizar os dados.\pause
\item Analisar os dados.\pause
\begin{itemize}[label={\ding{226}}]
\item Análise descritiva e exploratória (o que aconteceu, tem alguma informação inesperada?).\pause
\item Análise inferencial.\pause
\end{itemize}
\end{itemize}
    \end{block}
\end{frame}

\begin{frame}{Conceitos Importantes}
    \begin{block}{}
    \justifying
\begin{itemize}[label={\ding{228}}]
\item Pesquisa: Coleta de informações sobre uma característica de interesse de unidades de uma população, usando métodos e procedimentos bem definidos. Acompanhado da compilação dessas informações em uma forma resumida útil.\pause
\item Amostragem: Consiste em selecionar parte de uma população para observar, de modo que seja possível estimar alguma coisa sobre toda a população.\pause
\item Características desejáveis da amostra:\pause
\begin{itemize}[label={\ding{227}}]
\item Capacidade de generalização.\pause
\item Imparcialidade e representatividade.\pause
\item Capacidade de medir a precisão das estimativas.
\end{itemize}
\end{itemize}
    \end{block}
\end{frame}

\begin{frame}{Como podemos obter amostras adequadas?}
    \begin{block}{}
    \justifying
\tikzstyle{decision} = [diamond, draw, fill=blue!50]
\tikzstyle{mybox1} = [rectangle, minimum width=5cm, minimum height=1cm, text centered, text width=5cm, draw=black, fill=gray!50]
\tikzstyle{mybox2} = [rectangle, minimum width=5cm, minimum height=1cm, text centered, text width=5cm, draw=black, fill=green!50]
\tikzstyle{mybox3} = [rectangle, minimum width=5cm, minimum height=1cm, text centered, draw=black, fill=orange!50]
\tikzstyle{mybox4} = [rectangle, minimum width=5cm, minimum height=1cm, text centered, text width=5cm, draw=black, fill=green!30]
\tikzstyle{mybox5} = [rectangle, minimum width=5cm, minimum height=1cm, text centered, text width=5cm, draw=black, fill=orange!30]
\tikzstyle{line} = [draw, -stealth, thick]
\tikzstyle{elli}=[draw, ellipse, fill=red!50,minimum height=8mm, text width=5em, text centered]
\tikzstyle{block} = [draw, rectangle, fill=blue!50, text width=8em, text centered, minimum height=15mm, node distance=10em]
\begin{tikzpicture}
\node [mybox1] (n0) {Métodos de Amostragem};
\node [mybox2, left of=n0, xshift=-8em, yshift=-3.3em] (n1) {Probabilística};
\node [mybox3, right of=n0, xshift=8em, yshift=-3.3em] (n2) {Não Probabilística};\pause
\node[mybox4, below of=n1, yshift=-0.5em](n11){Aleatória Simples};\pause
\node[mybox4,below of=n1,yshift=-3em](n12){Sistemática};\pause
\node[mybox4,below of=n1,yshift=-5.5em](n13){Estratificada};\pause
\node[mybox4,below of=n1,yshift=-8em](n14){Por Conglomerados};\pause

\node[mybox5, below of=n2, yshift=-0.5em](n21){Por Conveniência};\pause
\node[mybox5,below of=n2,yshift=-3em](n22){Intencional};\pause
\node[mybox5,below of=n2,yshift=-5.5em](n23){Bola de Neve};\pause
\node[mybox5,below of=n2,yshift=-8em](n24){Por Cotas};
%arrows
\path [line] (n0) -- (n1);
\path [line] (n0) -- (n2);
\path [line] (n1) -- (n11);
\path [line] (n2) -- (n21);
%\draw [arrow] (n1) |-++ (n11);
%\path [line] (decision1) -| node[yshift=0.5em, xshift=10em] {yes} (process1);
%\path [line] (decision1) -| node[yshift=0.5em, xshift=-10em] {no} (process2);
\end{tikzpicture}
    \end{block}
\end{frame}

\section{Amostragem Aleatória Simples}
\begin{frame}{Amostragem Aleatória Simples}
\begin{block}{}
    \justifying
Procedimento do método:
\begin{description}
\item[(1)] Selecione uma unidade de $P$
(população) com probabilidade $1/N$.\pause
\item[(2)] Repita o passo \textbf{(1)} $n - 1$ vezes, sendo cada seleção independente das anteriores.
\end{description}
\end{block}
\end{frame}

\begin{frame}{}
\begin{block}{}
\justifying
\begin{minipage}[t]{0.45\textwidth}
\textbf{Vantagens}: 
\smallskip
\begin{itemize}[label={\ding{226}}]
    \item Simplicidade;
    \item Permite medir a precisão das estimativas;
\end{itemize}
\end{minipage}% % leave no gap
\begin{minipage}[t]{0.5\textwidth}
\textbf{Desvantagens}:
\smallskip
\begin{itemize}[label={\ding{227}}]
    \item Custo Elevado (Amostra Dispersa);
    \item Necessita de cadastro da população.
\end{itemize}
\end{minipage}
\end{block}
\end{frame}

\begin{frame}{Aplicações}
    \begin{block}{}
    \justifying
\begin{itemize}[label={\ding{228}}]
\item Qualquer situação em que se tenha um cadastro da população.\pause
\item Opinião dos alunos sobre alguma política estudantil.\pause
\item Clima organizacional de uma empresa. \pause
\item Satisfação dos alunos de Estatística básica com o ensino remoto.\pause
\item Satisfação dos clientes de uma rede bancária.\pause
\item Produtos retirados de um lote para inspeção.\pause
\item Sorteio de pessoas para serem mesários nas eleições.
\end{itemize}
    \end{block}
\end{frame}

\section{Dimensionamento de Uma Amostra}
\begin{frame}{Dimensionamento de Uma Amostra}
    \begin{block}{}
    \justifying
\begin{minipage}[t]{0.45\textwidth}
Na amostra, cada unidade é medida, sendo a média e o desvio padrão calculados
através das seguintes fórmulas, respectivamente:
\smallskip
$$\bar{x}=\dfrac{\displaystyle{\sum_{i=1}^{n}x_{i}}}{n}$$
\end{minipage}% % leave no gap
\begin{minipage}[t]{0.5\textwidth}
\begin{align*}
s&=\sqrt{\dfrac{\displaystyle{\sum_{i=1}^{n}\Big(x_{i}-\bar{x}\Big)^{2}}}{n-1}}\\
&=\sqrt{\dfrac{\displaystyle{\sum_{i=1}^{n}x_{i}-\frac{\Big(\displaystyle{\sum_{i=1}^{n}x_{i}}\Big)^{2}}{n}}}{n-1}}   
\end{align*}
\end{minipage}
    \end{block}
\end{frame}

\begin{frame}{}
\begin{block}{}
\justifying
Nas estimativas dos parâmetros de uma população, utilizando-se os resultados de uma amostra, há sempre um erro envolvido, denominado de erro de amostragem ou erro de estimativa, que aparece porque não se avaliou toda a população. Podendo ser estimado através da seguinte expressão:
\end{block}
\pause
\begin{block}{}
\begin{align*}
e=t_{\frac{\alpha}{2}}s(\bar{x})=t_{\frac{\alpha}{2}}\dfrac{s}{\sqrt{n}},
\end{align*}
em que,\pause
\begin{itemize}[label={\ding{228}}]
    \item $e$ é erro de estimativa da média da população com base nos resultados de uma amostra de tamanho n;\pause
    \item $t_{\frac{\alpha}{2}}$ é o quantil $\frac{\alpha}{2}$ na extremidade da cauda à direita da distribuição $t-$student com $n_{0}-1$ graus de liberdade;\pause
    \item $s$ é o desvio-padrão de uma amostra piloto de tamanho $n_{0}.$
\end{itemize}
\end{block}
\end{frame}

\begin{frame}{}
    \begin{block}{}
    \justifying
O erro de amostragem $(e)$ pode ser pré fixado de acordo com os objetivos do
estudo, permitindo assim, calcular o tamanho de uma amostra necessária para fornecer uma estimativa da média da população de acordo com um nível de significância $\alpha$, como segue:
\begin{align*}
    n=\Big(\dfrac{t_{\frac{\alpha}{2}s}}{e}\Big)^{2}
\end{align*}
    \end{block}
\end{frame}

\begin{frame}{Exemplo}
    \begin{block}{}
    \justifying
Para que o erro ao estimar o peso médio dos estudantes da Universidade
fosse de, no máximo $3,0$ kg, o dimensionamento da amostra poderia ser feito com base numa amostra piloto constituída por $n_0 = 10$ estudantes $(75, 82, 94, 66, 81, 77, 68, 98, 84, 80),$ por exemplo. Deste modo, com base em $\alpha=5\%$, $t_{2,5\%}(9) = 2,26$ e $s = 10,07,$ a amostra deveria ter, no mínimo:
\begin{align*}
    n=\Big(\dfrac{2,26\times 10,07}{3}\Big)^{2}=59
\end{align*}
    \end{block}
\end{frame}

\section{Amostragem Sistemática}
\begin{frame}{Amostragem Sistemática}
    \begin{block}{}
    \justifying
Procedimento do método:
Selecione uma unidade de partida $r$ ao
acaso entre 1 e K, com probabilidade
1/K.
2. Selecione cada K-ésima unidade do
cadastro a partir da primeira selecionada,
isto é, r + K, r + 2K, etc.
    \end{block}
\end{frame}

\begin{frame}{}
\begin{block}{}
\justifying
\begin{minipage}[t]{0.45\textwidth}
\textbf{Vantagens}: 
\smallskip
\begin{itemize}[label={\ding{226}}]
    \item Simplicidade;
    \item Fácil de Estimar quantidades populacionais (Exceto Va\-riân\-cia);
    \item Cadastro pode ser construído junto com a amostra.
\end{itemize}
\end{minipage}% % leave no gap
\begin{minipage}[t]{0.5\textwidth}
\textbf{Desvantagens}:
\smallskip
\begin{itemize}[label={\ding{227}}]
    \item Custo Elevado (Amostra Dispersa);
    \item Difícil para estimar a precisão da estimação;
    \item A Periodicidade no cadastro pode impactar nas estimativas.
\end{itemize}
\end{minipage}
\end{block}
\end{frame}

\begin{frame}{Aplicações}
    \begin{block}{}
    \justifying
\begin{itemize}[label={\ding{228}}]
\item Avaliação da qualidade de peças em uma linha de produção.\pause
\item Avaliação de lotes de produtos que chegam em sequência.\pause
\item Pesquisas de boca de urna.\pause
\item Pode substituir a amostragem aleatória simples.\pause
\item Plantas em um pomar ou lavoura.\pause
\item Animais que passam por um corredor.\pause
\item Pessoas em um teatro/cinema.
\end{itemize}
    \end{block}
\end{frame}


\begin{frame}{Exemplo}
    \begin{block}{}
    \justifying
A amostra sistemática apresenta características parecidas com a amostra
aleatória simples, porém por um processo mais rápido e mais simples. Por exemplo, se for retirada uma amostra de 1.000 fichas de uma população de 5.000 fichas, pode-se retirar sistematicamente, uma ficha a cada cinco fichas (5000/1000 = 5).
    \end{block}
\end{frame}

\section{Amostragem Aleatória Estratificada}
\begin{frame}{Amostragem Aleatória Estratificada}
    \begin{block}{}
    \justifying
Quando a população for heterogênea, não se deve usar a amostra aleatória
simples ou a amostragem Sistemática, devido à baixa precisão das estimativas obtidas. Nesta situação, deve-se dividir a população em subpopulações de forma que dento das subpopulações haja homogeneidade.
    \end{block}
\end{frame}

\begin{frame}{Amostragem Aleatória Estratificada}
    \begin{block}{}
    \justifying
Procedimento do método:
\begin{description}
    \item[1.~]Particione a população $U$ em $H$ grupos disjuntos e homogêneos, chamados de estratos.\pause
    \item[2.~]Selecione uma amostra dentro de cada um dos estratos, independentemente.\pause
    \item[3.~]A amostra a ser pesquisada é a união das amostras selecionadas nos estratos.\pause
\end{description}
Dentro de cada estrato pode ser necessário usar diferentes métodos de
coleta e/ou amostragem.
    \end{block}
\end{frame}

\begin{frame}{Aplicações}
    \begin{block}{}
    \justifying
\begin{itemize}[label={\ding{228}}]
    \item Sempre que a população puder ser estratificada.\pause
    \item Pesquisas eleitorais.\pause
    \item Clima organizacional em diferentes setores de uma empresa.\pause
    \item Pesquisas de opinião em populações fechadas e estratificadas. Ex. Na UFV podemos tomar como estratos os diferentes departamentos ou cursos.\pause
    \item Amostragem de árvores por talhão na floresta.\pause
    \item Amostragem de animais por rebanho.
\end{itemize}
    \end{block}
\end{frame}

\begin{frame}{Algumas medidas relacionadas aos estratos}
    \begin{block}{}
    \justifying
Considerando que os h estratos estejam devidamente organizados, pode-se
considerar a seguinte notação:
\begin{itemize}[label={\ding{228}}]
    \item $N_{h}:$ Número de elementos da população no estrato $h$;\pause
    \item $n_{h}:$ Número de elementos da amostra no estrato $h$;\pause
    \item $N=\displaystyle{\sum_{h=1}^{H}N_{h}}:$ Tamanho da população;\pause
    \item $n=\displaystyle{\sum_{h=1}^{H}n_{h}}:$ Tamanho da amostra;
\end{itemize}
 \end{block}
\end{frame}

\begin{frame}{Algumas medidas relacionadas aos estratos}
    \begin{block}{}
    \justifying
Em cada estrato, trabalha-se como se o processo envolvesse uma amostra
aleatória simples. Assim, para o estrato $h,$ o estimador da média populacional $\mu_{h}$ e da variância populacional $\sigma_{h}^{2}$ são:
\begin{minipage}[t]{0.45\textwidth}
\smallskip
$$\bar{x}_{h}=\dfrac{\displaystyle{\sum_{i=1}^{n_h}x_{hi}}}{n_{h}}$$
\end{minipage}
\begin{minipage}[t]{0.5\textwidth}
\begin{align*}
s_{h}^{2}&=\dfrac{\displaystyle{\sum_{i=1}^{n_h}\Big(x_{hi}-\bar{x}_{h}\Big)^{2}}}{n_{h}-1}\\
&=\dfrac{\displaystyle{\sum_{i=1}^{n_h}x_{hi}-\frac{\Big(\displaystyle{\sum_{i=1}^{n_{h}}x_{hi}}\Big)^{2}}{n_{h}}}}{n_{h}-1}  
\end{align*}
\end{minipage}
    \end{block}
\end{frame}

\begin{frame}{Algumas medidas relacionadas aos estratos}
\begin{block}{}
\justifying
O estimador da média da população $\mu$, chamada de média estratificada, é obtido ponderando-se as médias dos estratos, pelo número de elementos do estrato, ou seja:
\begin{align*}
\bar{x}_{est}=\dfrac{\displaystyle{\sum_{h=1}^{H}N_{h}\bar{x}_{h}}}{N}
\end{align*}
\end{block}
\nocite{Apostila}
\end{frame}

\begin{frame}%[allowframebreaks]
\frametitle{\bf Referências}
\printbibliography
\end{frame}


\end{document}
